\section{Séance 5 : Coûts dans l’entreprise}


\subsection{Formules}



\begin{itemize}
	\item CT: coût totaux
	\item CV: coûts variables
	\item CF: coûts fixes
\end{itemize}
$$CT=CV+CF$$


\begin{itemize}
	\item CMoT: coûts moyens totaux
	\item CMoV: coûts moyens variables
	\item CMoF: coûts moyens fixes
\end{itemize}
$$CMoT = \frac{CT}{Q}$$
$$CMoV = \frac{CV}{Q}$$
$$CMoF = \frac{CF}{Q}$$



\begin{itemize}
	\item CM : coût marginaux
\end{itemize}
$$CM  = \frac{dCT}{dQ}$$
$$CT = \int CM dQ + CF$$



\begin{itemize}
	\item PT: produit total
	\item PMo: produit moyen
	\item PM: produit marginal
\end{itemize}
$$PMo = \frac{PT}{Q}$$
$$PM = \frac{dPT}{dQ}$$



\begin{itemize}
	\item PT: recette totale
	\item PMo: recette marginale
	\item PM: produit marginal
\end{itemize}
$$RT = P.Q$$
$$RM  = \frac{dRT}{dQ}$$



\subsection{Profit}



$$\Pi=RT-CT=(RM-CM).Q$$



\subsection{Maximisation du profit}



$$Optimum = \frac{d\Pi}{dQ} = 0 \Leftrightarrow \frac{dRT}{dQ} - \frac{dCT}{dQ} = 0 \Leftrightarrow RM - CM = 0 \Leftrightarrow RM = CM$$